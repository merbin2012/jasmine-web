\documentclass[a4paper,twocolumn,10pt]{article}
\usepackage{usenix,epsfig,endnotes}
\usepackage{listings}
\usepackage{xspace}

\newcommand{\jasmine}{{\textsc{Jasmine}}\xspace}

\def\denseitems{
  \itemsep1pt plus1pt minus1pt
  \parsep0pt plus0pt
  \parskip0pt\topsep0pt}

\begin{document}

%don't want date printed
\date{}

%make title bold and 14 pt font (Latex default is non-bold, 16 pt)
\title{\Large \bf Specification of \jasmine}
\author{
{\rm Wanchun Li}\\
Georgia Institute of Technology\\
wli7@gatech.edu
}

\maketitle
\pagestyle{plain}

%%-------------------------------------------------------------------------
\section{Introduction}
\jasmine (\underline{JA}vaScript \underline{M}anipulation and \underline{In}strumentation at \underline{E}xecution-time) is tool for run-time JavaScript instrumentation and code manipulation. This document is the specification of \jasmine, including requirements, system design, and API design.

The goal is to instrument code at run-time for two benefits: to handle dynamism of JavaScript and to avoid a proxy.\footnote{An instrumentation proxy complicates the deployment, increases the costs of maintenance, and may introduce performance overhead under heavy workload.}
%%-------------------------------------------------------------------------

\section{Requirements}
This section describes requirements of run-time instrumentation and requirements of run-time code manipulation.
\subsection{Requirements of Execution-time Instrumentation}
Instrumentation point lists
\begin{itemize}\denseitems
  \item Function entry/exit
  \item Basic Block entry/exit
  \item Branch entry/exit
  \item Loop entry/exit
  \item Statement before/after

\noindent and
  \item data-flow alike analysis. It is incomplete analysis before of eval() or innerHTML)
  \begin{itemize}\denseitems
    \item parameter trace
    \item def-use of a particular variable
  \end{itemize}
\end{itemize}


\subsection{Requirements of Execution-time Code Manipulation}
libraries. Dynamic generated code from third party. Improve program Examples
\begin{itemize}\denseitems
    \item global variables
    \item iframe pollutes namespace
    \item string concate
    \item JSLint for more
\end{itemize}

Rewrite code not using eval() but the following code
\begin{quote}
    function hello(){\\
		console.log('hello outside');\\
	}	
	hello();
	var scr = document.createElement('script');
	scr.text = 'var t = function(){console.log("inner hello")}';
	document.getElementsByTagName('head')[0].appendChild(scr);
	hello = t;
	hello();
\end{quote}



\section{Execution Mode}
\subsection{Silent Mode}
The user specifies the portion of the program to instrument: the src and the function.

\subsection{Interactive Mode}
No instrumentation is executed until an event handler is trigger.

\section{Design}
\subsection{Instrumentation Points}
  \begin{itemize}\denseitems
    \item func\_entry/exit
    \begin{quote}
    /*function*/
    \end{quote}

    \item bbl\_entry/exit
    \begin{quote}
    /*basic block*/
    \end{quote}

    \item brc\_entry/exit
    \begin{quote}
    /*branch*/
    \end{quote}

    \item lp\_entry/exit
    \begin{quote}
    /*loop*/
    \end{quote}

    \item stat\_before/after
    \begin{quote}
    /*statement*/
    \end{quote}

  \end{itemize}
\subsection{Silent Mode}
\textbf{Probe}\\ probe = {uri, func, inst\_points, inst\_code}
\begin{quote}
uri: source of js;\\
func: name of the function to be instrumented\\
inst\_points: instrument point\\
inst\_code: instrument code\\
\end{quote}
\textbf{Instrumentation implementation}\\
Rewrite the source code into another file that overwrites the original methods. Reload the rewritten source code.
Algorithm consisting of two steps:
\begin{enumerate}
  \item all script sources that have functions to be instrumented register to the window object
  \item window.onload = the following sequence of actions
  \begin{enumerate}
    \item rewrite code to insert instrumentations
    \item load rewritten code
    \item check whether all rewritten code is loaded
    \item if not, wait; otherwise, finishing onload.
  \end{enumerate}
\end{enumerate}

\subsection{Interactive Mode}
\textbf{Probe}\\ probe = (evt, ele, uri, func, inst\_points, inst\_code)
\begin{quote}
evt: event fired
ele: element on which an event fired
uri: source of js;
func: name of the function to be instrumented
inst\_points: instrument point
inst\_code: instrument code
\end{quote}

Need to register eventListeners.\\

Using eval() to execute rewritten code. 

Separated implementation from instrumenting eval (See Section~\ref{sec_relatedIssues}, Issue~\ref{eval}).

\section{Related Issues}\label{sec_relatedIssues}
\begin{enumerate}\denseitems
    \item Dynamic code injection
        \begin{enumerate}\denseitems
            \item if the developer has the knowledge of the source code, instrument the source before injection
            \item if the developer does not have (e.g., legacy code, code from other domain), a message model monitoring code's altering (i.e., to check the change of script objects in every certain time)
        \end{enumerate}


    \item eval()~\label{eval}: rewritten. eval is rewritable by changing function behavior. So rewrite the string parameter.

    \item register eventListeners: parsing source code to build a event listeners storage list. Instrument the event handlers.

    \item top-level code instrumentation: proxy

\end{enumerate}




\end{document} 